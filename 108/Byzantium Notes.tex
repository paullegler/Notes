\documentclass{article}
\usepackage{enumerate}

 
\begin{document}

\textbf{Lecture 1: 8/24/2017} \\
Midterm, Final each worth 30 percent \\
Two 3-5 page papers worth 20 percent \\ \\
\textbf{Byzantine}: Eastern half of the Roman Empire after the fall of Roman that never fell to the Barbarians. They saw themselves as Romans, not Byzantines - those descended from this to this day (Greeks and Greek Orthodox) still call themselves Romans - means someone from the city of Constantinople \\
\textbf{Barbarian}: anyone who speaks something other than Greek - only an evaluation of disliking anything not like us \\ 
\textbf{Greek Orthrodox} coined in 19th century with rise of Nationalism - up until 19th/20th century Ottomans called themselves Romans and had a Greek education \\
\textbf{Hermonis Wolf??}: influential in how we divide up Byzantine history \\
\textbf{Byzas}: founded the city of Byzantium coming from Megara - St. Paul in 1st century AD stopped here \\
\textbf{Constantinople}: Constantine renamed Byzantium after himself so it would be his city - founded as an explicitly Christian capital - this helped move from Jewish sect to imperial religion. Moving the capital isn't bizarre/whimsical - makes sense \\
\textbf{Diocletian}: soldier-emperor - predecessor to Constantine - under him the Eastern part of the empire started to become the center and yield more taxes and the biggest organized army opposition was in Persia. He divided the empire into four different tetrarchs and each ruled from different capital (2 junior and 2 senior emporers). He also introduced economic system of 16 years. He wanted to move away from Roman traditions - towards an absolute monarchy - elaborate court ceremony introduced from Iran \\
\textbf{proskynesis}: prostrating in front of ruler - introducing practices from the East and South \\
\textbf{dominus}: he made himself absolute ruler away from Roman practice - term senator loses meaning it had earlier although the concept is still there - advice \\
Alexander the Great conquering the east heavily Hellenized the East. Most important function is to collect taxes - keep this in place and also keep language \\
Once Diocletian died, the other tetrarchs fought and Constantine remained standing and he chose capital \\
Why is it Roman?
\begin{itemize}
\item Roman Legal System: Byzantine empire run through the Roman legal system, although there was some new legislation - only gotten rid of in 1948 although some Orthodox Christian in Lebanon, Israel, Egypt in 1950s
\item Roman Imperial Ideal: loyalty to the office of the emperor - doesn't guarantee personal loyalty but instead the system
\item Territorial Nucleus remained largely the same until 1262 after the fourth crusade
\end{itemize}
\textbf{Byzantine History}: driven by Reformation period (Protestant) effort to convert oriental christians as well as Byzantium as a model for absoluteness - Louis the 14th in france looking for models on which to found his own. The French monarchy admired the organized bureaucratic system that allowed tax collection which is also why it's looked down on now \\ 
\begin{itemize}
\item 284 accession of Diocletian: made lots of political reforms
\item	324 Constantine?s monocracy, decision to found Constantinople: this took a few years to reorganize the new capital
\item	330 inauguration of Constantinople
\item	395 final division of the Roman empire into Eastern and Western
\item	476 fall of the Western Roman empire
\item	565 death of Justinian I: important 6th century emperor
\item	610 accession of Heraclius
\item	717 accession of Leo III
\end{itemize}

Other division (how the professor divines Byzantine period):
\begin{itemize}
\item Early Byzantine period: 324-610 (Constantine I to the fall of emperor Phocas): also called late antiquity or early Christian period (actually ends late sixth/early seventh century) - not the fall of the Roman but a lot of change such as a new religion. Peter Brown is the father of late antiquity but he has opposition who says his term is too general
\item	Middle Byzantine period: 610-1081 (Heraclius to Nikephoros Botaneiates): Heraclius loses to the Arabs the territory of Egypt and Syria  which had a lot of economic importance 
\item	Late Byzantine period: 1081-1453 (Alexios I to the fall of Constantinople to the Ottomans) Alexios I brings a lot of administrative changes
\end{itemize}
\textbf{Direct source}: no human mediation, minutes records from meetings, writings from an emporer \\
\textbf{Indirect source}: someone writes a narrative at the time - we need these to put the pieces together\\
\end{document}
