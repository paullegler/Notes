\documentclass{article}

\begin{document}

\tableofcontents

\newpage

\section{What is Anthropology}
\textbf{Etymology}: Comes from the ancient Greek words anthropos (human) and logos (logic) \\
\textbf{Logos}: "Word" often translated as Reason, Science, Disciplined Knowledge. Becomes central to religious modes of though (Gospel of John). Aristotle talked about logos, ethos, and pathos
\textbf{4 Fields of Anthropology in the U.S.}:  Physical Anthropology, Cultural (Social) Anthropology, Archeology, Linguistic Anthropology

\subsection{Renaissance Anthropology}
\begin{itemize}
\item They looked to the heavenly, cosmic forms, especially the stars (astrology)
\item Manilius wrote about how the signs corresponded to human conditions, natures, and types
\item Looking for correspondences between the human form and the celestial form
\end{itemize}
Renaissance elites also turned to the diversity of things in the world looking for the correspondences, especially the exotic, mysterious, frightening, and monstrous - wonder and curiosity. They formed \textbf{curiosity cabinets} by assembling exotic stuff and trying to draw ties. \\ \\
\textbf{Michel Focucault}: drawing from his professor, the idea was that we aren't correcting errors throughout human history, but instead the \textbf{episteme} (way of reasoning in a given moment) can only be understood separate from every other episteme. He said thinking looks like natural history, no longer looking for secret correspondences - but focused on forms and how to order them - order is fixed in time and did not change. \\
\textbf{Ibn Khaldun}: 14th century North African scholar who was rediscovered by Europe in the 19th century - he is sometimes taught as a harbinger of Modern social science, as the first modern social scientist - how do we hold ourselves together as a group

\section{The 19th Century Revolution in Time}
Charles Lyell was a Scottish Victorian geologist who wrote and influential book, Principles of Geology. He argues the the world was much older than 6000 years as the Bible said. This awareness of deep time was experienced by many in the 1830s as revolutionary. He was a devout Christian, troubled with the implications of his research on fossils, which showed that human were much older than the Bible suggested. \\
This becomes central to the study of the animal/human and would influence Darwin and others like Edward Tyler. Tyler spoke of \textbf{The Comparative Method} which said we must compare the savages (Africans, Indigenous Americans, etc.) with Europeans - this was a key clue to the origins of white british people. The dominant story is looking at a past group and trying to understand how it led to the current state. 

\subsection{A New Idea: Race}
Georges Cuvier (not an evolutionist) 18th to early 19th century, humans were understood to be divided into several immutable races.Cuvier, French scholar, argued for three major races. He didn't argue the way that Adam and Eve developed into 3 races, focused on aesthetic beauty, saying his white people were more beautiful. The biological difference between people hadn't existed before - the difference could be religious like Jewish \\
\textbf{Monogenism}: all human races came from the same ancestors \\
\textbf{Polygenism}: human races all came from different origins. Louis Agassiz was a Swiss scholar in the 19th century. They argued the races couldn't have come from the same origin, but it didn't survive modern biology - he was an evolutionist he just thought there were three seperate lines. \\ 
\textbf{Physical Anthropology}: emerged to infer where races fir on an evolutionary ladder. People became concerned about race mixing. People though mixing these three different species could be eugenically dangerous.

\subsection{A Different Anthropology: Franz Boas and "Culture"}
Culture is a certain way of thinking or of organizing problems. You can point out that race is not true really as a feature of biology, but is believed to be true. Culture is much less obvious. \\
\textbf{Culture}: What people do that is handed down to them as children

\section{Culture}
Culture is inherited through learning from schooling and people around you. Can also be thought of as artifacts or as sophistication \\
Culture emerged as a key concept of anthropology in the US at the end of the 19th century - associated with Franz Boas. Emerged in the context of differentiation between distinct racial types and ranking them: \textbf{physical anthropology} which differentiated between types based upon physical measurements of the body where Northern Europeans understood themselves to be the best \\ \\
Until the mid 19th century it referred to something the elites had

\subsection{Anthropometry}
Emerged across several new scientific fields in the late 19th century - one of these was \textbf{criminology} a new way to identify and keep records - were certain races pre-disposed to be criminals or were criminals a race (based on physical measurements) \\
\textbf{Colonial anthropometry}: imperial states used anthropometry to define and distinguish their colonized subjects in Asia and Africa as well as in some cases to look for traces of criminal disposition

\subsection{Boas}
Argued that much of human collective behavior, what people in groups did in common, was not only or primarily a matter of inborn bio, that is of race, but was in addition or rather learned behavior that formed part of the ethos or culture of the group - Nurture over Nature. \\
He brought the German idea of \textbf{Kultur} into American anthropology - each \textbf{Volk} (people) has its own unique culture and anthropologist studies this beauty and aesthetics \\ 
Boas was trained in part as a geographer - they take culture as a totality, as a form or a pattern or a configuration but to tunk about how that configuration was tied to an environment \\
Robert Lowie: culture are things of shreds and patches \\
Ruth Benedict: cultures each have unique configurations or patterns: her most famous book was called Patterns of Culture

\subsection{Matthew Arnold (Industrial England)}
Argued that people tend to do what their parents and people around them do (including the people in their class)  - culture was a matter of cultivating one's best self, of transcending the unthinking and selfish dictates of one's class - he said one should break out of one's and class and care and transform themselves - this is the opposite of the Boas viewpoint

\subsection{Edward Tyler: Wholism}
British anthropologist - wrote that culture is the complex whole which includes knowledge, belief, art, law, morals ... that are acquired by man as a member of society - Nurture \\ 
He differs from Arnold because Arnold thinks you must alienate from the group

\subsection{German Idealism}
These ideas of culture drew from German Idealism: \textbf{Bildung} is something you must do to have culture (like Arnold's) you must wrestle and improve yourself to become something beautiful. \textbf{Kultur}: there are things shared by a group like language and dance that allow a group to express itself

\section{Evolution vs Diffusion}
The fact that there are pyramids in both Egypt and Central America can be explained in both. Evolution says there is a way in which kings want monuments etc. but diffusion says no way that's possible they came from the same people and the idea diffused \\ \\ 
\textbf{Diffusion}: Certain traits, fragments of the specific patterns of unique cultures could diffuse between neighboring groups. Boas said that colonial pressure would diffuse away native cultures - he told his students to document these rich cultures that were disappearing because so much value was dying (salvage) \\  \\ 
\textbf{Salvage} Diffusionist pressure is pushing away cultures, so we must save them before they are gone - this eventually became a dirty word because it put down the native cultures as inferior - also you want to save our culture but what about us. Boas also said you must go back a step and imagine what it was like one step before they came.


\section{Margaret Mead: Coming of Age in Somoa}
\subsection{My initial comments from reading}
\begin{itemize}
\item She really emphasizes how they are "primitive" people
\item They seem to live care free
\item The fact that the children flow from relative to relative aids the care-free environment Mead believes in - it might actually indicate other things though like a lack of structure and order
\item Monogamy less stressed than it is in Europe at the time - a lot of adultery and informal sex, and simple divorce process
\item Women can not have a higher title than their husbands
\item Mead argues that adolescent girl's lives aren't really stressful as it is a slow process filled with casual sex and a slow coming of age. They have the simple goal of having relations with guys and then marrying one in her village and having kids.
\item First key difference is a lack of specialization of feeling
\item Their children are more well adjusted - they don't have to make as many choices
\item They are eased into birth, death, and sex early on and don't feel fear or anything towards them
\end{itemize} 

\subsection{Background}
Somoa became a valuable harbor and stopping/fueling point for trade ships in the 19th century. Britain, German, and US made territorial claims on Somoa - Somoan elites would make alliances with them. 1889 - battle between German and American ships over control. 1899 - Somoa split between Germans and Americans. \\ \\ 
The ethrnographic present: she erases all things she understands to not be a part of native culture 

\subsection{The Historical Method}
Find a culture less influenced by European cultures - she find a part of the island least Westernized to imagine what things were like before the diffusion of Western forms but people are now Protestant and there is a global economy and focus on only the traditional things - this is known as the \textbf{Ethnographic Present}. They saw the world as made up of ethnic group, whether this be races or cultures. The hope was to get to some theory that could generalize about the development of culture \\ \\ 
\textbf{Not in My Village}: preventing generalizations by showing a counterexample - In Somoa its different - this goes against the theory that all teens are stressed and fight with their parents about sex 

\subsection{G. Stanley Hall and Adolescence}
He wrote an influential book called Adolescence in 1907 that argued that youth was a distinct phase of life; we now take it for granted that kids go though this "storm and stress" (German romantic idea tied to the importance of \textbf{Bildung} \\ \\ 
\textbf{Moral Panic}: Mead says that maybe all these experts are coming together with an idea that is diffusing but its all smoke and just hormones in response to the pastors criticizing the youth of the day losing their values

\subsection{Culture as Patterns of Life}
Groups have worked out solutions to classic human problems like how to live together, survive, reproduce, and eat. She says "our" own culture (Indo-European culture - "Aryan race" came out of Sanskirt - trying to distance themselves from Hebrew) is a set of patterns that we shouldn't take for granted. Each pattern has aesthetic value. \\  \\ 
\textbf{How does one study a pattern?} \\
Spending time immersed in the cutlure (ethnography/fieldwork). She tries to give a sense of every day life from dawn until dusk and dawn again. She says what is heard, seen, sensed - the actors in her story are both human and non-human (bird, roosters, etc.) to paint the full picture. She also gives a sense of social structure  by kinship and marriage (villages/households). Most important principal in the village is age. Although the youth live a flexible life where they can move around between households - always a variety of relations of people who can care for each other \\ 
The \textbf{Taupo} (ceremonial princess/virgin) demonstrate fame/honor of the village/household. Their behavior is monitored very closely. (Mead was made one interestingly)

\subsection{Normal and Pathological}
Every culture selects a certain range of biological possibility (a given set of norms). Depending on a culture's normal, its abnormal (or pathological) would differ accordingly. Empire of Temperament - country selecting its traits over time \\ \\ 
Mead found variety of New Ginea cultures with varying aggression in men and women to make this point (two Gaussian curves society selects where the mean is) \\ \\ 
Each culture selects certain traits as normal and others as pathological (\textbf{Culture and Personality School}) - human beings may have a common biology but all these cultures came from different child rearing (later generations of Boas thought)

\subsection{Non-controversy vs Local Controversy}
Despite all the new/challenging material, it was not very controversial. Mead confirmed a story about the hyper-sexual Polynesian culture - whether that is a good thing/sign of relaxtion or bad/sinful/heathen thing \\ \\ 
Over time, Somoa became more Christians and many came to resent her book because it was not who they were any more or ever were - she's lying. Somoa became so famous in anthro, another non-Somoan anthropologist Derek Freeman did a restudy. He went to the same islands and he was also made a chief and spent a lot of time with him. These men described a different Somoa - one in which daughters are kept close at hand to make sure they are moral people. \\ \\ 
\textbf{Derek Freeman} goes farther and says that the entire Boas project is wrong - teenagers do go through stressful times objectively - took on Nature vs Nurture and took the side of Nature strongly. He became a celebrity. On his second trip back he found a girl who had talked to Margaret Mead who said she lied to Margaret Mead...get PWNED...they resisted her power to force them to reveal their lives to her questions. But this woman had been a taupo - had very restricted lives - so Mead didn't really rely on her

\subsection{Mead vs Freeman}
Position 1: Mead - evidence in her book - culture is real and informs behavior \\ 
Position 2: Freeman - evidence in his book - anthropologists underestimated importance of biology - challenging racial hierarchy \\
Position 3: They were both right a point in the following ways:
\begin{itemize}
\item History: they were there at different times in a changing world - two different moments in time - both useful
\item Situated Knowledge: he is talking to older men about what young girls and she is talking to young girls about what young girls do. But the vantage of the chiefs still matters and we must bring them together (knowledge depends on vantage point)
\item Anthropology itself had changed - the categories of nature and nurture just weren't helpful - the biology is out of date and the idea of culture didn't fit any more
\end{itemize}

\section{Franz Boas: Instability of Human Types}
\subsection{My initial comments from reading}
\begin{itemize}
\item He thinks there are clear differences (physically) between the races
\item He argues that these differences can change over time however - very interested in migrants and the way their children develop
\end{itemize}

\subsection{Historical Antrhopology for Boas}
He looks at the physical body in combination with cultural norms, material culture, and language These four fields come to dominate US anthropology. \\ He is always anxious about generalization though \\ \\
In 1911 he makes two claims: \\ 
The story was not a process of evolution where groups are fixed on it, but rather the more that society and culture evolve, the more the physical body will change
\begin{itemize}
\item nutritional/environmental factors will transform the body (like head size) - \textbf{plasticity} of anthropometric race
\item humans are more or less domesticated - difference between Europeans and southern African bushmen is like that between a domesticated animal and a non-domesticated animal 
\end{itemize}
Boas says there are three bodily effects of domestication on animals:
\begin{itemize}
\item Nutrition and use of the body
\item Selection (I want fluffy dog)
\item Crossing (Breeding)
\end{itemize}
The normal condition of humans is relatively easy access to food supplies. Boas argues that humans select sexual partners based on cultureal distinctions like being part of the right class and not anthropomorphic distinctions like being really tall (endogamy - marry someone like me race/religion) - exogamy is the opposite and the dividing line is incest

\section{The Body as an Ethnological Object}
\textbf{Settler Society}: arrival of Europeans, migrations from Asia, forced migration of Africans, displacement of Native Americans - was also very diverse \\
\textbf{Nature} became a concept of race as fixed and hierarchical and one dominant way race was made powerful claim on expanding diversity was by measurement of the human body (anthropometry)

\subsection{Challenging 19th Century Race}
Claims of nature replaced with biology (different claims of nature) - anthropology questioned the idea of a racial hierarchy \\ 
Every society is only given certain set of tools to think and reason with - not all given the same thing - this is close to the idea of episteme (way of getting at truth) - this came to be called \textbf{the social construction} argument: any understand of nature/anthropos is a particular model or store people tell about their world; it is not reality or even a mirror of reality but an effort to understand

\subsection{Cultural Ptterns and Norms Create Bodily Forms - Culture Makes Nature}
Cultural norms of diet create physical conditions of weight. 

\section{Marcel Mauss: Techniques of the Body}
\subsection{My initial comments from reading}
\begin{itemize}
\item Talks about habits like swimming which differ among peoples and generations
\item "echnique an action which is effective and traditional (and you will see that in this it is no different from a magical, religious or symbolic action"
\item "Two things were immediately apparent given the notion of techniques of the body: they are divided and vary by sex and by age."
\item "Training, like the assembly of a machine, is the search for, the acquisition of an efficiency"
\end{itemize}
\subsection{Background}
French scholar part of the new school of sociology that emerged at the end of the 19th century \\
Anthropologists study the other (usually marginal others on a power gradient), and sociologists study us. Anthropologists also focus on the specific while sociologists focus on the general. Anthropologists primary method is ethnography while for sociologists they divide quantitative survey research and ethnography \\ \\
\textbf{Ethnography}: description and documentation, the individual case \\ 
\textbf{Ethnology}: comparison and generalization/analysis, careful generalization and distinction \\
One needs multiple examples to generalize (ethnography) \\
Mauss thinks its useful to study things that others don't care about or categorize - science advances at the fringe 

\subsection{Habitus}
The acquired ability and faculty of Aristotle - not just a habit.  \\
Mauss says introduces psychology  - a habit is never simply a psychological state - it becomes part of the physical body \\
He emphasizes education and the way people use their bodies usually come from imitation, which some people are better at than others. \\ \\ 
\textbf{Education} not only the power of others - authority: it is people who have as well as prestige, who models oneself on older relations and teachers, the coach/trainer, the cynosure (role model), the sone, the tool \\ 
\textbf{Confidence}: to go beyond the biological resistance or limit, using "magical" properties of tool or magic word can allow one to overcome something (e.g. hunter with spear) \\
Techniques must be capable (known) and effective (help achieve the goal) - they are not rites 

\section{W.E.B. Du Bois, The Souls of Black Folk}
\subsection{My initial comments from reading}
He feels divided - he is an American and a Negro and has a hard time reconciling these. He makes a comment about sociologists simply counting their number of children or something and says this is not important - he wants to focus largely on developing education of blacks - artistry mostly
\subsection{Comments from section}
Hurston writes stories within stories and doesn't even mark it sometimes \\
\textbf{Fiction}: Latin fictio - making - sense-making practice - the fact that these authors are black studying blacks leads them away from objective positivist thought \\
\textbf{Positivism}: Kant that social laws can figured out objectively from experiments and find laws that will always be followed (logical positivism more about logic) \\
\textbf{Double Consciousness}: divide between Negro and white world - people don't know what it is like to be a problem - seeing oneself through the other world \\
Not helpful to think about if witchcraft exists but to think about the impact it has \\ 
When he studied in Germany, he felt more human and less seen only as a Negro

\subsection{The Philadelphia Negro}
He wants to study the facts by doing house to house interviews/quantitative data and arrive at a valid generalization. He records numbers of households with number of criminals in black and white households trying to "number the problem" \\ \\
He is always fair in his takeaways even from a racist reviewers viewpoint - however this guy is still very concerned about mixing "inferior" with "superior" blood and W.E.B. didn't address this issue at all.

\subsection{Double Consciousness}
How does it feel to be a problem? - he feels a part of two different worlds - looking at one's self through the eyes of others - measure one's soul by the tape of a world that looks on in contempt and pity - his goal is to find some oneness that merges these worlds \\ \\
Du Bois living at a time of African men getting vote for the first time and there are Africa Americans going to university for the first time - the absolute division between black and white he can notice when he goes to teach in the deep south \\ \\
For Boas: Is America one culture with a split or is it two entirely separate cultures? \\
\textbf{Society}: a group of people who work together as a collective and don't fall apart - they have some solidarity created by institutions to prevent us from descending into "Hobbesian Warre"

\subsection{Durkheim}
Much of Durkheim's work was concerned with how societies could maintain their integrity and coherence in modernity

\section{Robert Redfield: The Folk Society and Culture}
For Boas a pattern/culture is a homogeneous whole by itself but Redfield argues that we must see anthropology from the viewpoint of modernization \\
\subsection{Modern}
"The whole is society" - doesn't divide into sections with different cultures \\
Redfield is a positivist anthropologist as he wants to come to some kind of general conclusion \\
\textbf{Anthropology vs Sociology} - antrho studies primitive cultures (pre-modern) especially scared (myth, ritual, religion), status, and stasis, in a homogenous society while sociology studies complex urbanized societies especially secular aspects in a heterogenous, mobile, mordern world \\
\textbf{Folk/Peasant Society}: modernizing \\
We must keep in mind that Redfield is from a city - he actually argues that the people are fundamentally different to justify why the difference between anthro and soc exist \\ 
Anthropology predicated on change - they create cultural static other about primitive societies to contrast with modernity - primitive are "still" susceptible to myth unlike us

\section{Systems}
Anthropology debate shifts to a focus on modernization with the peasant being the intermediary between primitive and urban. The urban societies are organized around science - you don't simply do what your parents do - you do what experiments have shown - people get a job based on achievement and this is what value is tied to (\textbf{pattern variable}) \\
\subsection{Feedback Loops}
When people get sick (the stimulus), there is a sensor to detect this - if this obstructs the role then the sensor may trigger an administrative event requiring some action
\section{Structuralism}
The central debate arises between existentialism and structuralism. \\
\subsection{Existentialism}
Viewed as almost a fad that only some people get to be involved in (French cafe for rich people) \\
Fundamental question: more than just what is the meaning of life - it is how does one live a meaningful life?? \\
You can look at cultural norms and follow these or study biology/society and try to understand through the sciences (if we only knew more we could know the meaning of life - but Heidegger says this can't tell us how to live). \\
Martin Heidegger (Nazi) asks the ontological question what it means \textbf{for me} to be. How can a Nazi be asking these questions? \\
One must be responsible for one's actions \\
The Bhagavag Gita says you must complete your assigned role deliberately and with devotion to God. 
\subsection{Philology}
The study of written (often ancient) documents - they study languages - increasingly this became a comparative project in the 19th century - how languages seem to change over time \\
\textbf{Historical Linguistics} using comparative method - two languages compared and you can tell they have a common ancestor by breaking down into phonetic units \\
Perian and Northern India also showed relatedness by this method - but Hebrew and Arabic, East Asain and Sub-Saharan Africa didn't appear to ave the same ancestor as Indo-European \\
\textbf{Ferdinand De Saussure}: Swiss scholar, focus on Sanskrit, who said Historical linguistics only looked at structures historically over time, and did not try to understand the structure of a language at a given time \\
\textbf{Structural Linguistics}: analysis of how language interacts with culture at a given time \\
Parole/Individual: how a given speaker uses the language. But Langue/Collective: the rules and structures that organize a given language 
\subsection{Tabla Rasa vs Plato}
The debate is empiricism vs. idealism. \\
Tabla Rasa: the mind is empty and perceives items and classifies them over time. \\
Plato: there are things innate in the mind and we are able to project them on reality \\
Or maybe somewhere in between: the answer lies in the nervous system, not in the mind or experience, but the nervous system is built by nature and experience \\ \\
Science has a different relationship with totality than myth - mythic thought presume you understand the whole Cosmos, while science does not.\\
Nature has a limited number of procedures that reappear at levels - bound to reality \\
\subsection{LS Structuralism}
Emerges mid-century - draws on concept of system from cybernetics. Has a focus on binary code allowing for an analysis of relations between opposed elements; not about function. He doesn't make a claim about what people should do but rather focus on how thinking works - myth/science binary \\
Parsons focuses on feedback loops that maintain order of different roles within the system so it functions - stresses maintenance of normative behavior unless experts advise otherwise - traditional/modern \\ \\
\textbf{Writing} does divide two groups of people. The first is concrete thinking 















\end{document}
