\documentclass{article}

\begin{document}
\section{What is Anthropology}
\textbf{Etymology}: Comes from the ancient Greek words anthropos (human) and logos (logic) \\
\textbf{Logos}: "Word" often translated as Reason, Science, Disciplined Knowledge. Becomes central to religious modes of though (Gospel of John). Aristotle talked about logos, ethos, and pathos
\textbf{4 Fields of Anthropology in the U.S.}:  Physical Anthropology, Cultural (Social) Anthropology, Archeology, Linguistic Anthropology

\subsection{Renaissance Anthropology}
\begin{itemize}
\item They looked to the heavenly, cosmic forms, especially the stars (astrology)
\item Manilius wrote about how the signs corresponded to human conditions, natures, and types
\item Looking for correspondences between the human form and the celestial form
\end{itemize}
Renaissance elites also turned to the diversity of things in the world looking for the correspondences, especially the exotic, mysterious, frightening, and monstrous - wonder and curiosity. They formed \textbf{curiosity cabinets} by assembling exotic stuff and trying to draw ties. \\ \\
\textbf{Michel Focucault}: drawing from his professor, the idea was that we aren't correcting errors throughout human history, but instead the \textbf{episteme} (way of reasoning in a given moment) can only be understood separate from every other episteme. He said thinking looks like natural history, no longer looking for secret correspondences - but focused on forms and how to order them - order is fixed in time and did not change. \\
\textbf{Ibn Khaldun}: 14th century North African scholar who was rediscovered by Europe in the 19th century - he is sometimes taught as a harbinger of Modern social science, as the first modern social scientist - how do we hold ourselves together as a group

\section{The 19th Century Revolution in Time}
Charles Lyell was a Scottish Victorian geologist who wrote and influential book, Principles of Geology. He argues the the world was much older than 6000 years as the Bible said. This awareness of deep time was experienced by many in the 1830s as revolutionary. He was a devout Christian, troubled with the implications of his research on fossils, which showed that human were much older than the Bible suggested. \\
This becomes central to the study of the animal/human and would influence Darwin and others like Edward Tyler. Tyler spoke of \textbf{The Comparative Method} which said we must compare the savages (Africans, Indigenous Americans, etc.) with Europeans - this was a key clue to the origins of white british people. The dominant story is looking at a past group and trying to understand how it led to the current state. 

\subsection{A New Idea: Race}
Georges Cuvier (not an evolutionist) 18th to early 19th century, humans were understood to be divided into several immutable races.Cuvier, French scholar, argued for three major races. He didn't argue the way that Adam and Eve developed into 3 races, focused on aesthetic beauty, saying his white people were more beautiful. The biological difference between people hadn't existed before - the difference could be religious like Jewish \\
\textbf{Monogenism}: all human races came from the same ancestors \\
\textbf{Polygenism}: human races all came from different origins. Louis Agassiz was a Swiss scholar in the 19th century. They argued the races couldn't have come from the same origin, but it didn't survive modern biology - he was an evolutionist he just thought there were three seperate lines. \\ 
\textbf{Physical Anthropology}: emerged to infer where races fir on an evolutionary ladder. People became concerned about race mixing. People though mixing these three different species could be eugenically dangerous.

\subsection{A Different Anthropology: Franz Boas and "Culture"}
Culture is a certain way of thinking or of organizing problems. You can point out that race is not true really as a feature of biology, but is believed to be true. Culture is much less obvious. \\
\textbf{Culture}: What people do that is handed down to them as children

\section{Culture}
Culture is inherited through learning from schooling and people around you. Can also be thought of as artifacts or as sophistication \\
Culture emerged as a key concept of anthropology in the US at the end of the 19th century - associated with Franz Boas. Emerged in the context of differentiation between distinct racial types and ranking them: \textbf{physical anthropology} which differentiated between types based upon physical measurements of the body where Northern Europeans understood themselves to be the best \\ \\
Until the mid 19th century it referred to something the elites had

\subsection{Anthropometry}
Emerged across several new scientific fields in the late 19th century - one of these was \textbf{criminology} a new way to identify and keep records - were certain races pre-disposed to be criminals or were criminals a race (based on physical measurements) \\
\textbf{Colonial anthropometry}: imperial states used anthropometry to define and distinguish their colonized subjects in Asia and Africa as well as in some cases to look for traces of criminal disposition

\subsection{Boas}
Argued that much of human collective behavior, what people in groups did in common, was not only or primarily a matter of inborn bio, that is of race, but was in addition or rather learned behavior that formed part of the ethos or culture of the group - Nurture over Nature. \\
He brought the German idea of \textbf{Kultur} into American anthropology - each \textbf{Volk} (people) has its own unique culture and anthropologist studies this beauty and aesthetics \\ 
Boas was trained in part as a geographer - they take culture as a totality, as a form or a pattern or a configuration but to tunk about how that configuration was tied to an environment \\
Robert Lowie: culture are things of shreds and patches \\
Ruth Benedict: cultures each have unique configurations or patterns: her most famous book was called Patterns of Culture

\subsection{Matthew Arnold (Industrial England)}
Argued that people tend to do what their parents and people around them do (including the people in their class)  - culture was a matter of cultivating one's best self, of transcending the unthinking and selfish dictates of one's class - he said one should break out of one's and class and care and transform themselves - this is the opposite of the Boas viewpoint

\subsection{Edward Tyler: Wholism}
British anthropologist - wrote that culture is the complex whole which includes knowledge, belief, art, law, morals ... that are acquired by man as a member of society - Nurture \\ 
He differs from Arnold because Arnold thinks you must alienate from the group

\subsection{German Idealism}
These ideas of culture drew from German Idealism: \textbf{Bildung} is something you must do to have culture (like Arnold's) you must wrestle and improve yourself to become something beautiful. \textbf{Kultur}: there are things shared by a group like language and dance that allow a group to express itself

\section{Evolution vs Diffusion}
The fact that there are pyramids in both Egypt and Central America can be explained in both. Evolution says there is a way in which kings want monuments etc. but diffusion says no way that's possible they came from the same people and the idea diffused \\ \\ 
\textbf{Diffusion}: Certain traits, fragments of the specific patterns of unique cultures could diffuse between neighboring groups. Boas said that colonial pressure would diffuse away native cultures - he told his students to document these rich cultures that were disappearing because so much value was dying (salvage) \\  \\ 
\textbf{Salvage} Diffusionist pressure is pushing away cultures, so we must save them before they are gone - this eventually became a dirty word because it put down the native cultures as inferior - also you want to save our culture but what about us. Boas also said you must go back a step and imagine what it was like one step before they came.


\section{Margaret Mead: Coming of Age in Somoa}
\subsection{My comments from initial reading}
\begin{itemize}
\item She really emphasizes how they are "primitive" people
\item They seem to live care free
\item The fact that the children flow from relative to relative aids the care-free environment Mead believes in - it might actually indicate other things though like a lack of structure and order
\item Monogamy less stressed than it is in Europe at the time - a lot of adultery and informal sex, and simple divorce process
\item Women can not have a higher title than their husbands
\item Mead argues that adolescent girl's lives aren't really stressful as it is a slow process filled with casual sex and a slow coming of age. They have the simple goal of having relations with guys and then marrying one in her village and having kids.
\item First key difference is a lack of specialization of feeling
\item Their children are more well adjusted - they don't have to make as many choices
\item They are eased into birth, death, and sex early on and don't feel fear or anything towards them 

\subsection{The Historical Method}
Find a culture less influenced by European cultures - she find a part of the island least Westernized to imagine what things were like before the diffusion of Western forms but people are now Protestant and there is a global economy and focus on only the traditional things - this is known as the \textbf{Ethnographic Present}. They saw the world as made up of ethnic group, whether this be races or cultures. The hope was to get to some theory that could generalize about the development of culture \\ \\ 
\textbf{Not in My Village}: preventing generalizations by showing a counterexample - In Somoa its different - this goes against the theory that all teens are stressed and fight with their parents about sex 

\subsection{G. Stanley Hall and Adolescence}
He wrote an influential book called Adolescence in 1907 that argued that youth was a distinct phase of life; we now take it for granted that kids go though this "storm and stress" (German romantic idea tied to the importance of \textbf{Bildung} \\ \\ 
\textbf{Moral Panic}: Mead says that maybe all these experts are coming together with an idea that is diffusing but its all smoke and just hormones in response to the pastors criticizing the youth of the day losing their values

\end{itemize}

\end{document}
