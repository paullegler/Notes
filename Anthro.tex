\documentclass{article}

\begin{document}
\section{What is Anthropology}
\textbf{Etymology}: Comes from the ancient Greek words anthropos (human) and logos (logic) \\
\textbf{Logos}: "Word" often translated as Reason, Science, Disciplined Knowledge. Becomes central to religious modes of though (Gospel of John). Aristotle talked about logos, ethos, and pathos
\textbf{4 Fields of Anthropology in the U.S.}:  Physical Anthropology, Cultural (Social) Anthropology, Archeology, Linguistic Anthropology

\subsection{Renaissance Anthropology}
\begin{itemize}
\item They looked to the heavenly, cosmic forms, especially the stars (astrology)
\item Manilius wrote about how the signs corresponded to human conditions, natures, and types
\item Looking for correspondences between the human form and the celestial form
\end{itemize}
Renaissance elites also turned to the diversity of things in the world looking for the correspondences, especially the exotic, mysterious, frightening, and monstrous - wonder and curiosity. They formed \textbf{curiosity cabinets} by assembling exotic stuff and trying to draw ties. \\ \\
\textbf{Michel Focucault}: drawing from his professor, the idea was that we aren't correcting errors throughout human history, but instead the \textbf{episteme} (way of reasoning in a given moment) can only be understood separate from every other episteme. He said thinking looks like natural history, no longer looking for secret correspondences - but focused on forms and how to order them - order is fixed in time and did not change. \\
\textbf{Ibn Khaldun}: 14th century North African scholar who was rediscovered by Europe in the 19th century - he is sometimes taught as a harbinger of Modern social science, as the first modern social scientist - how do we hold ourselves together as a group

\section{The 19th Century Revolution in Time}
Charles Lyell was a Scottish Victorian geologist who wrote and influential book, Principles of Geology. He argues the the world was much older than 6000 years as the Bible said. This awareness of deep time was experienced by many in the 1830s as revolutionary. He was a devout Christian, troubled with the implications of his research on fossils, which showed that human were much older than the Bible suggested. \\
This becomes central to the study of the animal/human and would influence Darwin and others like Edward Tyler. Tyler spoke of \textbf{The Comparative Method} which said we must compare the savages (Africans, Indigenous Americans, etc.) with Europeans - this was a key clue to the origins of white british people. The dominant story is looking at a past group and trying to understand how it led to the current state. 

\subsection{A New Idea: Race}
Georges Cuvier (not an evolutionist) 18th to early 19th century, humans were understood to be divided into several immutable races.Cuvier, French scholar, argued for three major races. He didn't argue the way that Adam and Eve developed into 3 races, focused on aesthetic beauty, saying his white people were more beautiful. The biological difference between people hadn't existed before - the difference could be religious like Jewish \\
\textbf{Monogenism}: all human races came from the same ancestors \\
\textbf{Polygenism}: human races all came from different origins. Louis Agassiz was a Swiss scholar in the 19th century. They argued the races couldn't have come from the same origin, but it didn't survive modern biology - he was an evolutionist he just thought there were three seperate lines. \\ 
\textbf{Physical Anthropology}: emerged to infer where races fir on an evolutionary ladder. People became concerned about race mixing. People though mixing these three different species could be eugenically dangerous.

\subsection{A Different Anthropology: Franz Boas and "Culture"}
Culture is a certain way of thinking or of organizing problems. You can point out that race is not true really as a feature of biology, but is believed to be true. Culture is much less obvious. \\
\textbf{Culture}: What people do that is handed down to them as chi

\end{document}
